
As motivation, this section  offers the real-world goal that
sparked this work.  The Software Engineering
Institute (SEI: http://www.sei.cmu.edu/) at Carnegie
Mellon University is a Federally-Funded Research and
Development Center. As such, SEI works with many
software and software-intensive systems, which are
primarily undertaken for the United States
government (and the US Department of Defense in
particular) but also in commercial industry as
well. As a result of involvement in these programs,
the SEI is a storehouse for multiple repositories of
qualitative and quantitative data collected from
software development.
Developers and managers from across the country look
to SEI for explanations of what factors effect their
project (these explanations are used to manage their
current projects and well as propose methods on how
to better handle their future projects in a better
manner).  Also, SEI researchers may have a
voice in large-scale governmental policy decisions about
information technology.

 
{\em Fact sheets} are one tool used by the SEI for explaining its advice
on best practices and lessons learned
(with
traceability back to the data on which they are
based).  These
explanations are short reports (one to two pages)  which are intended
to give busy managers quick guidance for their
projects. In the case of quantitative data, these
explanations may contain some 2D plot showing how
one objective (e.g. defects) changes in response to
changes in one input variable (e.g. lines of code).

 
Since they are aimed at communicating with busy
professionals on specific points, fact sheets of
necessity represent tradeoffs made between
understandability and accuracy. That said,
overly simplistic fact sheets
(while approachable for a larger audience) run the risk of being
uninformative. For example, consider the use of 2D plots.
Even for single goal
reasoning such as defect reduction, these are
poorly characterized via one input variable. Studies with SE
data have compared models learned  $N =
1$  and $N > 1$ input variables. The models that used
more than one input performed better~\cite{me07b}.
Further, there are many recent SE research publications that propose multiple competing
goals for SE models; e.g.
\bi
\item
Build software {\em faster} using {\em less} effort with {\em fewer} bugs~\cite{elrawas10};
\item
Return defect predictors that find {\em most} defects in the {\em smallest} parts of the 
\cite{arisholm6}.
\ei
As we move from single goal to multiple-goal reasoning, the value of examining important factors in isolation using relatively simple plots
becomes even more questionable. The SBSE experience is that reasoning and
trading off between multiple goals is much more complex than browsing effects related
to a single isolated goal.

 
Additionally, given all the context variables that can be used to describe different
software projects, we recognize it is unlikely that any 
{\em one} report can explain {\em all} the effects
seen in all different kinds of software projects. Several reports in empirical
SE offer the same {\em locality effect}; i.e. models built from  {\em all}  data perform
differently, and often worse, than those learned from specific subsets~\cite{posnet11,betta12,me12d,yang13,emse12}.


How can we augment SEI's fact sheets such that business users can quickly read and understand:
\bi
\item
The space of effects in multiple dimensions of inputs?
\item
The responses of multiple objectives to changes in those inputs?
\item
And do so across the space of multiple contexts?
\ei
To handle this problem,
we propose printing succinct decision trees (generated by WICKED)
as part of the SEI fact sheet. Standard decision trees have leaves
that predict for a single class variable. WICKED's trees, on the other hand,
terminate in clusters that comment on multiple objectives. 
As described below, these trees can be used by domain users to generate
multiple explanations about their domain. Better yet, as shown by our experiments,
these small trees actually produce recommendations that are effective at optimizing
a domain across multiple objectives.
