
We have tested WICKED on numerous MOEA tasks
including the standard laboratory problems (DTLZ, Schaffer, Fonseca, etc) and found
it recommendations generated instances with objective scores competitive with
those generated by NSGA-II or SPEA2. 

Results from those standard lab problems
are rarely convincing or interesting to software project managers. Hence, we show
results from two business-level process models.
POM3~\cite{port08,me09j}  implements the Boehm and Turner model~\cite{port08,1204376,turner03}
of agile programming
where teams select tasks as they appear in the scrum backlog.
POM3 studies the implications of different ways to adjust task lists in the face of shifting priorities.
XOMO~\cite{me07f,me09a,me09e}
is four software process models from
the University of Southern California.
XOMO reports four-objective scores (which we will try to minimize):
project {\em risk};
development {\em effort} and {\em defects}; and total {\em months} of development.


\begin{figure}[b!]
\scriptsize

  \centering
    \begin{tabular}{|l|l|p{3in}|c|}
        \hline
        Short name &Decision             & Description         &Controllable                                        \\ \hline
        Cult&Culture              & Number (\%) of requirements that change. & yes \\\hline
        Crit&Criticality           & Requirements cost effect for safety critical systems. & yes\\\hline
        Crit.Mod&Criticality Modifier & Number of (\%) teams affected by criticality.   & yes           \\ \hline
        Init. Kn&Initial Known        & Number of (\%) initially known requirements.             & no     \\ \hline
        Inter-D&Inter-Dependency     & Number of (\%) requirements that have interdependencies.  Note that dependencies are requirements within
the {\em same} tree (of requirements), but interdependencies are requirements that live in {\em different} trees.   & no            \\\hline
        Dyna&Dynamism             & Rate of how often new requirements are made. & yes                    \\ \hline
        Size&Size            & Number of base requirements in the project.& no \\        \hline
        Plan&Plan                 & Prioritization Strategy (of requirements): 
        0= Cost Ascending;  1= Cost Descending; 2= Value Ascending; 3= Value Descending;
        4 = $\frac{Cost}{Value}$ Ascending.
%Note that a standard agile strategies use ``Value Descending'', i.e. plan=3~\cite{me09j}.
 & yes \\\hline
     T.Size&Team Size            & Number of personnel in each team   & yes                         \\ 
        \hline
    \end{tabular}
    \caption {List of Decisions used in POM3 (optimizers tune controllables, on right).}\label{fig:pom3decisions}
\end{figure}


\begin{figure}[!t]
\scriptsize
\begin{center}
    \begin{tabular}{r|p{1.2in}|p{1.2in}|p{1.2in}}
                     & POM3a                         & POM3b             &POM3c       \\ 
                             & A broad space\newline of projects. & Highly critical\newline small projects& Highly dynamic\newline large projects\\\hline
        Culture              & 0.10 $\leq x \leq$ 0.90       & 0.10 $\leq x \leq$ 0.90  & 0.50 $\leq x \leq$ 0.90  \\ 
        Criticality          & 0.82 $\leq x \leq$ 1.26       & 0.82 $\leq x \leq$ 1.26   & 0.82 $\leq x \leq$ 1.26  \\ 
        Criticality Modifier & 0.02 $\leq x \leq$ 0.10       & 0.80 $\leq x \leq$ 0.95 & 0.02 $\leq x \leq$ 0.08   \\ 
        Initial Known        & 0.40 $\leq x \leq$ 0.70       & 0.40 $\leq x \leq$ 0.70  & 0.20 $\leq x \leq$ 0.50  \\ 
        Inter-Dependency     & 0.0   $\leq x \leq$ 1.0       & 0.0   $\leq x \leq$ 1.0  & 0.0   $\leq x \leq$ 50.0 \\ 
        Dynamism             & 1.0   $\leq x \leq$ 50.0      & 1.0   $\leq x \leq$ 50.0  & 40.0   $\leq x \leq$ 50.0 \\ 
        Size                 & x $\in$ [3,10,30,100,300] & x $\in$ [3, 10, 30]     & x $\in$ [30, 100, 300]   \\ 
        Team Size            & 1.0 $\leq x \leq$ 44.0        & 1.0 $\leq x \leq$ 44.0  & 20.0 $\leq x \leq$ 44.0    \\ 
        Plan                 & 0 $\leq x \leq$ 4             & 0 $\leq x \leq$ 4    & 0 $\leq x \leq$ 4       
\end{tabular}
\end{center}

\caption{Three classes of projects studied using POM3. }\label{fig:POM3abcd}
\end{figure}


\subsection{POM3}
Turner and Boehm say that the agile
management challenge is to strike a balance between the three objectives
of {\em completion rates},
{\em idle rates},
 and {\em overall cost} of a project.
In the agile world, projects terminate after achieving a 
{\em completion rate} of   $(X<100)$\% of its required tasks.
Team members  become
{\em idle} if forced to wait for a yet-to-be-finished task from other teams. 
To lower {\em idle rate} and  
increase {\em completion rate}, management can hire staff- but this
can increase  {\em overall cost}.


\begin{wrapfigure}{r}{2in}
{\scriptsize
\begin{tabular}{l|r@{:~}p{1.2in}|}\cline{2-3}
scale   &prec & have we done this before?\\
factors &flex & development flexibility \\
(exponentially        &resl & any risk resolution activities?\\
 decrease       &team &  team cohesion\\
 effort)       &pmat & process maturity\\\hline
upper  &acap & analyst capability\\
(linearly       &pcap & programmer capability\\
 decrease      &pcon & programmer continuity\\
effort)       &aexp &  analyst experience\\
       &pexp &  programmer experience\\
       &ltex &  language and tool experience\\
       &tool &  tool use\\
       &site &  multiple site development\\
       &sced & length of schedule   \\\hline
lower &rely &    required reliability  \\
(linearly      &data &   secondary memory  storage requirements\\
increase      &cplx &  program complexity\\
effort)      &ruse &  software reuse\\
      &docu &   documentation requirements\\
      &time &   runtime pressure\\
      &stor &   main memory requirements\\
     &pvol &    platform volatility  \\\cline{2-3}
\end{tabular}}
\caption{
XOMO model decisions.}\label{fig:emsf2}
\end{wrapfigure}
When optimizing POM3, we seek 
changes to the controllables of \fig{pom3decisions} that maximize {\em completion rate}
while minimizing  {\em cost} and {\em idle}ness. 
To make this task more realistic, we run POM
for three different kinds of software projects, denoted POM3a, POM3b, POM3c
shown in \fig{POM3abcd}. We make no claim that these three projects cover X\% of all
software projects-- rather, our point here is that POM3 can handle different kinds of models.


\subsection{XOMO}


The XOMO model enables an exploration of competing factors 
within software projects. Ideally, management decisions can minimize
all of  {\em months}, {\em
  effort}, {\em defects} and {\em risk}. However, there are many trade-offs to be considered.
For
example.
increasing
software reliability   {\em reduces} the
  number of added defects while {\em increasing} the 
software development effort. For another example,
better documentation can improve team communication and {\em decrease} the number of introduced defects.
However, such increased documentation {\em increases} the development effort.

\begin{figure}[!t]
\scriptsize
\begin{tabular}{l@{~}|l@{~}r@{~}r@{~}}
project&&low&high\\\hline
FL:&rely&3&5\\
JPL flight &data&2&3\\
software&cplx&3&6\\
&time&3&4\\
&stor&3&4\\
&acap&3&5\\
&apex&2&5\\
&pcap&3&5\\
&plex&1&4\\
&ltex&1&4\\
&pmat&2&3\\
&tool&2&2\\
&sced&3&3\\
&KSLOC&7&418
\end{tabular}~~~~~~~~~~~~~~~~~~~~~~\begin{tabular}{l@{~}|l@{~}r@{~}r@{~}}
project&&low&high\\\hline

GR:&rely&1&42\\
JPL ground &data&2&3\\
software&cplx&1&4\\
&time&3&4\\
&stor&3&4\\
&acap&3&5\\
&apex&2&5\\
&pcap&3&5\\
&plex&1&4\\
&ltex&1&4\\
&pmat&2&3\\
&tool&2&2\\
&sced&3&3\\
&KSLOC&11&392
\end{tabular}~~~~~~~~~~~~~~~~~~~~~~\begin{tabular}{l@{~}|l@{~}r@{~}r@{~}}
project&&low&high\\\hline
O2:&prec&3&5\\
Orbital Space&pmat&4&5\\
Place guidance&docu&3&4\\
navigation and&ltex&2&5\\
control (v2)&sced&2&4\\
&flex&3&3\\
&resl&4&4\\
&time&3&3\\
&stor&3&3\\
&data&4&4\\
&pvol&3&3\\
&reuse&4&4\\
&...&&\\
&KSLOC&75&125
\end{tabular}
\caption{Three case studies used in XOMO.}\label{fig:xomocases}
\end{figure}

To explore those trade offs, XOMO uses the inputs of \fig{emsf2}
to drive  four models. The {\em effort} model predicts for 
``development months'' where one month
is 152 work hours by one developer (and includes development and management hours): 
\begin{equation}\label{eq:cocII}
\mathit{effort}=a\prod_i EM_i *\mathit{KLOC}^{b+0.01\sum_j SF_j}
\end{equation}
Here, {\em EM,SF} denote the effort multipliers and scale
factors and
 $a,b$ are the {\em local calibration} parameters which in COCOMO-II
have default values of 2.94 and 0.91.

The XOMO {\em defect}
model~\cite{boehm00b}  assumes that
certain variable settings {\em add} defects while
others may {\em subtract} (and the final defect count is the number of additions, less the
number of subtractions). 

Also the {\em months} model predicts for total development time and   can be used to determine staffing levels
for a software project. For example, if {\em effort}=200  
and {\em months}=10, then this project needs 
$\frac{200}{10} =20$
developers.

Lastly, the  {\em risk} model comments on how sets of management decisions decrease the 
odds of successfully completing a project. For example suppose a manager demands
{\em more}  reliability ({\em rely}) while  {\em decreasing} analyst capability ({\em acap}).
Such a project is ``risky'' since it means the manager is demanding more reliability from less skilled analysts.
The XOMO {\em risk} model contains dozens of rules that trigger on each
such ``risky'' combinations of decisions~\cite{madachy97}. 

In the following, we run three different case studies
of  XOMO: one for each of the NASA projects described in \fig{xomocases}.
